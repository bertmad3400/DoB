\documentclass[a4paper, 8pt, oneside]{article} % A4 paper size, default 11pt font size and oneside for equal margins

\newcommand{\plogo}{\fbox{$\mathcal{BM}$}}

\usepackage{appendix}

\usepackage{graphicx} % For images
\graphicspath{ {images/} }

\usepackage{scrextend} % For referencing the same footnote multiple times

\usepackage{parskip} % For properly having seperated lines

\usepackage{dblfnote} % For having footnotes in multiple collumns
\DFNalwaysdouble % for this example

\usepackage{microtype}

\usepackage[danish]{babel} % For danish name of things like abstract
\usepackage[utf8]{inputenc} % Required for inputting international characters
\usepackage[T1]{fontenc} % Output font encoding for international characters
%\usepackage{fouriernc} % Use the New Century Schoolbook font

% For adding colored text
\usepackage[dvipsnames]{xcolor}

\usepackage{listings}

\definecolor{codegreen}{rgb}{0,0.6,0}
\definecolor{codegray}{rgb}{0.5,0.5,0.5}
\definecolor{codepurple}{rgb}{0.58,0,0.82}
\definecolor{backcolour}{rgb}{0.95,0.95,0.96}

\lstdefinestyle{mystyle}{
    backgroundcolor=\color{backcolour},
    commentstyle=\color{codegreen},
    keywordstyle=\color{magenta},
    numberstyle=\tiny\color{codegray},
    stringstyle=\color{codepurple},
    basicstyle=\ttfamily\footnotesize,
    breakatwhitespace=false,
    breaklines=true,
    captionpos=b,
    keepspaces=true,
    numbers=left,
    numbersep=5pt,
    showspaces=false,
    showstringspaces=false,
    showtabs=false,
    tabsize=2
}

\usepackage[
citestyle=authoryear-ibid,
bibstyle=apa,
]{biblatex}
\addbibresource{SRU4.bib}

\usepackage[margin=0.7in]{geometry}

\usepackage {hyperref}

%----------------------------------------------------------------------------------------
%	TITLE PAGE
%----------------------------------------------------------------------------------------

\lstset{style=mystyle}

\begin{document}

\newgeometry{left=3cm,bottom=3cm}
\begin{titlepage} % Suppresses headers and footers on the title page

	\centering % Centre everything on the title page

	\scshape % Use small caps for all text on the title page

	\vspace*{\baselineskip} % White space at the top of the page

	%------------------------------------------------
	%	Title
	%------------------------------------------------

	\rule{\textwidth}{1.6pt}\vspace*{-\baselineskip}\vspace*{2pt} % Thick horizontal rule
	\rule{\textwidth}{0.4pt} % Thin horizontal rule

	\vspace{0.75\baselineskip} % Whitespace above the title

	{\LARGE DEMOKRATI\\ OG\\ } % Title

	\vspace{0.75\baselineskip} % Whitespace below the title

	{\LARGE BLOCKCHAIN}

	\vspace{0.75\baselineskip} % Whitespace below the title

	\rule{\textwidth}{0.4pt}\vspace*{-\baselineskip}\vspace{3.2pt} % Thin horizontal rule
	\rule{\textwidth}{1.6pt} % Thick horizontal rule

	\vspace{2\baselineskip} % Whitespace after the title block

	%------------------------------------------------
	%	Subtitle
	%------------------------------------------------
	
	Et dyk ned i hvordan vi kan anvender moderne teknologi at sikre vores demokratiske processer og gøre dem billigere, simplere og mere gennemskuelige.\\

	\vspace*{1\baselineskip} % Whitespace under the subtitle

	Programmering B og Matematik A% Subtitle or further description
	
	\vspace*{3\baselineskip} % Whitespace under the subtitle
	
	%------------------------------------------------
	%	Editor(s)
	%------------------------------------------------
	
	Skrevet af
	
	\vspace{0.5\baselineskip} % Whitespace before the editors
	
	{\scshape\Large Bertram Madsen - 5.BT} % Editor list
	
	\vspace{0.5\baselineskip} % Whitespace below the editor list
	
	\textit{Bagsværd kostskole og\\ Gymnasium} % Editor affiliation

	\vspace{0.3\baselineskip} % Whitespace below the editor list

	\textit{Antal anslag: 20415} % Editor affiliation
	
	\vfill % Whitespace between editor names and publisher logo
	
	%------------------------------------------------
	%	Publisher
	%------------------------------------------------
	
	\plogo % Publisher logo
	
	\vspace{0.3\baselineskip} % Whitespace under the publisher logo
	
	2021 % Publication year
	
\end{titlepage}
%----------------------------------------------------------------------------------------

\newpage
\tableofcontents
\newpage

\newpage
\begin{abstract}
	En demokratisk samfundstruktur er noget vi ser spille absolut central rolle i langt de fleste vestlige samfund, og noget som - til en vis grad - lægger til grund for næsten alle beslutninger og ændringer der idag bliver fortaget på landsplan i langt de fleste lande i hele verden.\footcite{desilver_despite_nodate}. På trods af dette er demokratiske proccessor - som bl.a. eksemplificeret ved det danske folktingsvalg - ikke nogle processor som får særligt meget opmærksomhed, og har derfor i stor grad ikke fulgt med den teknologiske udvikling som ellers har præget resten af de vestlige samfund.\\
    
	Dette projekt arbejder med at forsøge at belyse udfordringerne ved traditionelle valg, både i form af dem vi ser udelukkende ved brug af klassiske metoder som i Danmark, samt delvist digitaliserede valg som vi ser dem f.eks. i USA. I forbindelse med dette opstilles der et potentielt løsningsforslag ved brug af moderne blockchain teknologier, der redegøres for matematiken bag som gør det muligt og der diskuteres potentielle udfordringer ved denne type løsning og hvordan disse muligvis kan imødekommes.
\end{abstract}

\newpage

\section{Introduktion}
\label{Introduktion}
Demokrati er en grundsten i vores samfund som vi kender dem i dag i Danmark og mange andre vestlige lande. Derfor er derfor også kritisk for vores samfund at demokratiske processer som de danske folketingsvalg og de amerikanske føderale valg er sikre og derved repræsentative for befolkningen, for at sikre at vores demokrati resulterer i noget som gavner flertallet i befolkningen. I lang tid har den primære måleskala for hvor repræsentative og derigennem hvor sunde demokratierne er været valgdetagelsen for befolkning, da den giver en god indikation af hvorvidt at valgresultatet egentligt er repræsentativt for befolkning.

Desværre viser både valgdeltagelsen for de danske folkeafstemninger og kommunalvalg samt de amerikanske føderale valg til tider en bekymrende lav valgdeltagelse. Selvom de danske folketingsvalg siden 1953 har bibeholdt hvad mange ville beskrive som en sund stemmeprocent på over 80\%\footcite{clemmensen_folketingsvalgene_nodate}, så har de danske folkeafstemninger haft en svingende valgdeltagelse på mellem cirka 50\% og 85\%\footcite{clemmensen_folkeafstemninger_1953}, det sidste kommunalvalg i 2021 en stadigt faldende valgdeltagelse på 67,2\%\footcite{november_2021_stemmeprocent_2021} og de føderale valg i USE var helt nede omkring 37.5\% tilbage i 2014\footcite{noauthor_voter_nodate}.

Men trods at valgdeltagelsen ved det føderale valg i USA i 2014 var så lav at mange ville argumentere for at den er grænsene til ikke længere rent faktisk at repræsentere den amerikanske befolkning var valgdeltagelsen ikke det eneste problem. Studier som "Who Votes Now?" skrevet af Jan E. Leighley og Jonathan Nagler\footcite{leighley_who_2013} og lignende har nemlig i nyere tid belyst nogle generelle tendenser, som gør det til en markant mere problematisk udfordring, nemlig at:

\begin{enumerate}
	\item Valgdeltagelsen er ikke bare lav, men også markant forskudt. Det er en tung overvægt af overklassen, og en klar underrepræsentation (relativt til antallet af stemmeberetigede) af den lave middelklasse og underklassen blandt de amerikanere der stemmer.
	\item Dem der stemmer er ikke repræsentative for befolkningen. Altså har dem som stemmer nogle holdninger og præference som er så markant anderledes end dem der ikke stemmer, at hvis valgdeltagelsen 
\end{enumerate}

Lignende danske studier har - trods den markant højere danske valgdeltagelse - på lignene vis belyst tendenser i det danske samfund hvor der er en skævvridning i valgdeltagelsen på tværs af diverse demografiske grupper\footcite{bhatti_hvem_2014}, og derfor handler denne opgave om følgende problemformulering:\\

\begin{center}
    \textit{\Large Hvad er årsagen til den lave valgdeltagelse i både Danmark og USA, hvordan kan vi arbejde på at højne denne og hvilke udfordringer og konsekvenser vil der være ved føromtalte løsningsforslag.}
\end{center}

\newpage

\section{Det nuværende problem}
De specifikke årsager til at valgdeltagelsen ikke er højere i Danmark i nyere tid er desværre ikke noget der er har været særligt mange studier på, og dem der har været er ofte baseret på antagelser og meget små stikprøvestørrelser. Disse placere desuden ofte skylden på en generel følelse af manglende indflydelse blandt vælgerne\footcite{noauthor_demokratiet_nodate} eller politikerlede\footcite{ejsing_stor_2015}, hvilket begge er faktorer som er svære at ændre på uden en markant, udemokratisk indgriben.

Amerikanske undersøgelser har tilgengæld vist et markant mere nuanceret billede. Et studie fra det amerikanske FiveThirtyEight - som laver dataindsamling og analyse - i samarbejde med det franske forskningsfirma Ipsos har blandt mere end 8.000 amerikanere fundet at den lave valgdeltagelse har mindre at gøre med de politiske forhold og mere at gøre med de praktiske\footcite{noauthor_why_2020}. Studiet har nemlig fundet at for et stor del amerikanere handler udfordringerne ved at stemme mindre om det politiske landskab og mere om dårlige praktiske forhold, som eksemplificerede ved at næsten 20\% rapportere at have ventet ved en stemmeboks i mere end en time, at mere end 10\% af dem som normalt ville stemme simpelthen missede deadlinen, og en lang række andre praktiske problemer. Ydermere har Ipsos i et andet studie bragt i Reuters fundet at 74\% af amerikanerne var bekymret for potentialet for organiseret valgsvindel i præsidentvalget for 2020, en grundlæggende manglende tillid til systemet der naturligvis kan være med til at sænke stemmeprocenten. Denne manglende tillid er desuden ifølge begge studier også noget som bliver forstærket af at vælgerne ikke er i stand til faktisk at verificere at deres stemme er blevet talt.

Sammenholder vi dette med data fra Danmarks Statistik som viser at der stadig er flere og flere danskere som brevstemmer\footcite{nortoft_stadig_nodate} bliver to tendenser klare: 1. Vælgerne er til tider udfordret af de praktiske omstændigheder, noget som sænker valgdeltagelsen, og 2. Vælgerne er bekymret for sikkerheden ved diverse valg. Sammenholder vi disse data kan vi altså lave en liste over udfordringer ved den klassiske form for valg skal løse:

\begin{enumerate}
	\item Ventetiden for at stemme skal reduceres. Dette vil gøre det langt mindre uoverskueligt faktisk at få stemt.
	\item Det skal være muligt at stemme ligegyldigt din fysiske lokation. Dette ville eliminere problemer som ikke at kunne finde stemmeboksen samt at skulle have fri for arbejde for at kunne stemme og ville samtidigt gøre det markant nemmere for vælgerne, da de ikke behøver at forlade husets trygge 4 rammer for at stemme.
	\item Systemet skal være sikkert og gennemskuelig på en måde hvor det er muligt at bevise at der ikke er blevet snydt med valget. Dette ville skabe en tilid blandt befolkningen til at valget faktisk er forgået retfærdigt.
\end{enumerate}

Derudover kan vi også lave en list over krav hvor det ville være fordelagtigt hvis løsningen gav mulighed for at...

\begin{enumerate}
	\item ...ændre sin stemme. Herved ville der være mindre press på når man stemte (især som førstegangs-vælger), da man altid ville kunne ændre det hvis man skifte holdning.
	\item ...benytte en decentraliseret model. Dette ville betyde at lige meget hvem der havde magten, ville de ikke være i stand til rykke ved valgresultatet.
	\item ...verificere at ens stemme var blevet registreret og talt. Dette ville baseret på tidligere nævnte studier sandsynligvis øge valgdeltagelsen. 
	\item ...følge med i valget live, som stemmerne rullede ind.
\end{enumerate}

Udover disse problemer er der også en del praktisk udfordringer associeret med sig. Det er (grundet landets størrelse) ikke nær så stort et problem her i Danmark, men i USA kan valgende godt vise sige at være et større projekt med en lang række praktiske udfordringer. Et estimat fra MIT placerer de årlige omkostninger for administrationen forbundet med de 2-årige føderale valg på et minimum af 2 milliarder dollars årligt. Derfor er det også centralt at vi udvikler en løsning der ikke bare opfylder de tidligere krav, men som også skalerbar og økonomisk bæredygtig.

\subsection{Nuværende løsninger}
Der er en række løsninger der i øjeblikket er blevet forsøg anvendt til at løse dele af tidligere omtalt problematik, men deres success har desværre været ganske begrænset. En af de nuværende løsning der i øjeblikket er blevet taget delvist i brug i dele af USA er den kontroversielle Voter ID lovgivning som kræver en form for identifikation (varierende fra stat til stat) for at stemme. Den forsøger at løse problemet hvor der sker valgsnyd på et individuelt plan, hvor et individ lykkedes med af flere omgange, ved at stemme på vegne af andre. Lovgivningen er som sagt kontroversiel, da der er mange politiske grupper og lobby organisationer der påstår at den forhindrer især visse demografiske grupper i at stemme, ved at gøre det så svært at det i praksis er umuligt. Der er dog endnu ikke entydig videnskabelig evidens som understøtter disse påstande, men der er heller endnu ikke videnskabelig evidens til at understøtte at Voter ID faktisk er effektivt, og selv hvis det er, arbejder det kun med en meget lille del af problematikken.

USA har også i en årrække forsøgt og lykkedes at implementere digital stemmebokse for at arbejde på den enorme udfordring det er at have et valg der dækker et land på mere end 300 millioner indbygger. Desværre har dette vist sig at medføre markante problemer i forhold til sikkerheden, og den internationalle konferrence for IT-sikkerhed og hacking DEFCON skabte allerde i 2018 overskrifter landet over da de demonstrerede at det var så nemt at hacke de digitale stemmebokse at børn bogstaveligt talt var i stand til det\footcite{hern_kids_2018}. Derfor er dette naturligvis heller ikke en bæredygtig løsning, og selvom mere moderne digitale stemmebokse er under udvikling, er der sandsynligvis stadig meget lang tid til at de rammer markedet\footcite{ng_darpas_nodate}.

\section{Nyt løsningsforslag}
Det løsningforslag denne opgave beskæftiger sig med er et der minder meget om de digitale stemmebokse fra USA og på samme tid er radikalt anderledes. Frem for at anvende centraliserede digitale systemer anvendes der i dette projekt et blockchain system - nærmere beskrevet i næste afsnit - som benytter sig af af matematik til at skabe en uforanderlig kæde af beskeder, hvor det er muligt at verificere afsenderen bag hver besked. Dette gøres ved brug af asymetrisk kryptering til at bevise afsenderen for hver besked, og hashingfunktioner til at verificere at kæden ikke er blevet modificeret.

Denne model ville ikke bare løse samtlige af vores problemer vedrørende de praktiske forhold omkring at stemme (da det vil være muligt stemme øjeblikkeligt lige meget din fysiske lokation), men grundet strukturen bag en blockchain er det også en 100\% transparent måde at føre valg på, hvor det er matematisk muligt at bevise at den er sikker, og muligt for vælgerne ikke bare at følge med i valget live, men også at verificere at deres stemme faktisk er blevet både registreret og talt. Ydermere er det også en måde at stemme på, hvor de praktisk udfordringer vedrørende at holde valget er stærkt minimeret (men dog ikke fuldstændigt elimineret - for flere detaljer refere til underafsnittet "PoW - en praktisk udfordring") og hvor at stemmerne kan optælles i løbet af få sekunder. Altså er det en model der løser alle de tidligere problemer, om end den skaber også et par nye udfordringer vi skal arbejde lidt med.

\newpage

\section{Et dyk ned i teorien bag}
Men selvom at en digitale valg opfylder mange af de tidligere krav, så opstiller det at stemme digitalt to nye krav nemligt at:

\begin{enumerate}
	\item Det skal være muligt at verificere hvilken vælger stemmen kommer fra.
	\item Det skal være umuligt at skrive historien af stemmer om, når først den er nedskrevet (eller gemt digitalt).
\end{enumerate}

Disse er nogle krav som en blockchain løser ved hjælp af to lidt abstrakte koncepter, 1. signerede transaktioner og 2. proof of work i kombination med blokke, som illustreret i den følgende model:\\

\begin{figure} [ht]
  \centering
  \includegraphics[width=0.95\textwidth]{blockchain.png}
	\caption{https://www.mdpi.com/1424-8220/21/17/5874}
  \label{fig:blockchain}
\end{figure}


Det simpleste element i en blockchain er hvad der ofte referres til som en "transaction", eller transaktion på Dansk. Navnet kommer fra kryptovalutaer, hvor hver transaktion svare til en bankoverførsel, men i dette system vil hver transaktion referer til en stemme udsendt af en vælger. Disse er på det simpleste niveau bare en datastruktur som indeholder et ID der unikt identificerer en vælger (i denne implementering vælgerens CPR-nummer), hvad de stemmer (f.eks. hvilket parti) samt en digital signatur der i forbindelse med en offentlige nøgler, knyttet til vælgerens ID kan bruges til at verificere at den eneste person der har været i stand til at afgive den stemme er vælgeren selv.

Disse transaktioner kan herefter samles i bundter af transaktioner kaldet blokke og når der er tilstrækkeligt med transaktioner i en blok, "forsegles" blokken ved en ressource intensiv process ofte referret til som proof of work. Ved at bruge hashing funktioner sikre man her at det tager lang tid at "forsegle" blokken, men når først den er forseglet vil enhver ændring af blokkene bryde forseglingen, og det er desuden nemt matematisk er verificere at forseglingen er intakt.

Sidst men ikke mindst har vi selve blockchain datastrukturen. Dette er - som navnet antyder - en kæde af blokke, hvor at hvis bare et enkelt led i kæden ændre, invalideres hele resten af kæden. Dette forgår ved at inden man forsegler hver blok inkluderes hashet af den tidligere blok i den nye blok. Dette betyder at det for hver blok er muligt at verificere at den tidligere blok ikke er blevet ændre, og derved er man i stand til at verificere hele kæden.

\subsection{Teorien bag hashing algoritmer}
Hashing funktioner er en funktion som tager en streng af data med en arbitrær længde ind i binær form og spytter en streng af data ud med en fast længde. Outputtet af en hashing funktion er per definition pseudo-tilfældig, hvilket betyder at outputtet ser tilfældigt ud, og det baseret på outputtet er (næsten) umuligt at finde det oprindelige input, men et givent input vil altid give det samme output. Derudover er sandsynligheden for at få det samme output fra to forskellige inputs (et begreb kendt som hash kollision) tæt på umulig (igen, per design). Dette gør hashing funktioner optimale til at verificere at indholdet af en datastruktur ikke har ændret sig, da den - lige meget datastrukturen - outputter et streng i et håndterbart og konsistent format, som er svær at forfalske.

Selve hashfunktionen anvendt i dette projekt (og mange andre som f.eks. bitcoin) er SHA256, da den har alle de egenskaber listet ovenover og på samme tid tager det en forudsigelig og konsistent tid at beregne et hash for et vilkårligt input. Hashingfunktioner som SHA256 (og andre kendte hashing funktioner som SHA1 og MD5) benytter sig af en Merkel-Damgård konstruktion, hvor inputtet først bliver paddet med noget padding der koder for længden af inputtet (Merkle–Damgård forstærkning), hvorefter den bliver kørt igennem en en-vejs kompressions algoritme. Det Ralph Merkle og Ivan Damgård beviste var at så længe at algoritmeerne bag padding og kompressionen var kryptografisk stærke og kollisions-resistente, ville en hashing funktion bygget på dem også være kryptografisk stærk og kollisions-resistant.

En en-vejs kompressions algoritme er som navnet antyder en funktion der er i stand til at komprimere data af en given størrelse kan i tilfældet af en Merkel-Damgård konstruktion indgå i hashing funktionen H således ud\footcite{thomsen_cryptographic_nodate}:

Først defineres en række konstanter og funktioner:

Clear text: M\\
Clear text size: N\\
Hash af clear text: h\\
Hashing funktion: H(M) = h\\

Herefter defineres komprimeringsalgoritmen som f, samt størrelsen af inputtet og outputtet, med følgende definition:

Input størrelse: I\\
Output størrelse: O\\

\[f\left(\{0,1\}^O, \{0,1\}^{I-O}\right) \rightarrow \{0,1\}^O\]

Nu hvor vores input er delt op i to, kan vi også definere to nye konstanter:

Input 1 størrelse: $s_1$\\
Input 2 størrelse: $s_2$\\

Herefter kan funktionen $pad_{s_2, r}$ for at tilføje padding defineres som følgende:

Givet inputtet, defineret tidligere som M, og input størrelsen, defineret tidligere som N, tilføj 1 '1' (sand) bit, og tilføje derefter $((- N - r - 1) \; mod\; s_2 )$ '0' (falsk) bits. Til sidst tilføjes en repræsentation af M med en størrelse r til M, hvorved vi har sikret at M kan deles op i n antal lige store dele som alle har størrelsen $s_2$. Denne tilføjes af en represæntation af M med en størrelse af r lægger godt nok en begrænsning på størrelsen af N (til at være $2^r - 1$), men i praksis er i denne grænse så høj at det ikke er problematisk, og denne tilføjelse gør algoritmen markant mere kollisions resistent (det kaldes Merkle-Damgård forstærkning).

For at anvende vores hashing funktion H kan vi altså gøre følgende:

\begin{enumerate}
	\item Lad $M_p$ være en paddet version af M, ved brug af tidligere beskrevet algoritme, og lad $N_p$ være størrelsen af denne
	\item Del $M_p$ op i t antal blokke $m_1 ... m_t$, hver med længden $s_2$
	\item Lad $h_0$ en start-tilstand af H
	\item Beregn $h_1$ til $h_t$ på følgende vis:
		\[h_i \leftarrow f(h_{i - 1}, m_i), \;\; \textrm{for} \; 1 \leq i \leq t\]
	\item Lad til sidst H(M) være lig $h_t$
\end{enumerate}

Herved har vi altså beskrevet hashing funktion baseret på Merkle-Damgård konstruktionen, og fordelen her er at så længe at f er kollisions resistent, er H det også. Hvad der specifikt forgår i f varierer fra algoritme til algoritme, men et simpelt proof of concept kunne se således ud, hvis man lader $I_1$ være det ene input, $I_2$ være det andet, og O være størrelsen af outputtet:

\[f(I_1, I_2, O) = (I_1 + I_2) \; \textrm{mod} \; 2^O\]

Ved brug af denne "smidder" vi så meget information væk ved brug af modulos operatoren at det bliver tæt på umuligt at genskabe den oprindelige betydning.

\section{Udfordringer ved nuværende løsning}
Selvom at en blockchain som koncept løser alle de tidligere problemer og opfylder de krav vi har opstillet, stiller den os over for nogle nye udfordringer, vi skal finde en måde at takle på hvis den skal have en chance for at blive anvendt til dette formål.

\subsection{Offentlige stemmer}
Den måske tydeligste udfordringer ved det nuværende system er at alle stemmer per design er offentlige. Når vi forsøger at udvikle en bedre løsning til vores eksisterende demokratier er dette naturligvis ikke et acceptabelt kompromis og derfor er det nødvendigt at finde end løsning på dette. Mulige løsninger inkludere at benytte et par asymetriske nøgler til at signere et andet par asymetriske nøgler som så bruges til at signere transaktionerne (om end dette ville kræve en centralisering af systemet), eller ved at benytte sig af "Blind Signing" som beskrevet af David Chaum i hans bog "Advances in Cryptology"\footcite{chaum_advances_2012}.

\subsection{PoW - en praktisk udfordring}
En af de største udfordringer vedrørende blockchains er at selv processen for at forsegle blokkene er per design ikke bare ressourceintensiv men også kritisk for netværket, hvorfor det er vigtigt - for at opretholde den decentraliserede struktur - at vi har en del noder på netværket, repræsenteret af vælgernes digitale enheder som mobiltelefoner og computerer. Dette kunne godt være problematisk, taget i betragtning af at et af de problemer vi forsøger at løse er en lav valgdeltagelse, da det ville yderligere komplicere processen for vælgerne.

En interresant løsning på dette problem kunne være at frem for at anvende en decentraliseret blockchain, at anvende en kæde af transaktioner. Dette kunne gøres ved at have en centraliseret enhed som samler stemmer ind, og lade tidspunktet for stemmen indgå i transaktionen. Herved opnår vi desværret ikke en decentraliseret model, men vi opnår stadig en som løser samtlige af de praktiske udfordringer, dette problem og som er 100\% sikker og transparent.

\section{Konklusion}
Blockchain teknologi er en interresant mulighed for at sikre digitale valg, og gøre dem på en måde som giver borgeren mulighed for at følge med i valget, og garantere sikkerheden af det ved brug af matematiske funktioner. Det giver ydermere muligheden for at afholde flere valg end tidligere, da det er nemmere både at arrangere valgene og at stemme, men disse fordele er desværre ikke nogen som kommer uden udfordringer.

Det kunne være interresant i fremtiden - som tidligere nævnt - at undersøge muligheden for i stedet for en blockchain at benytte sige af en kæde af rene transaktioner i stedet. Her kunne man ofre fordelene ved et decentraliseret system for at løse udfordringen med at have noder på netværket til at håndtere proof of work arbejde, hvilket stadig ville bibeholde mange af fordelene ved denne metode frem for klassiske måde at stemme på.

Skulle dette implementeres på en sikker og effektiv måde og løses problemet med at alle stemmer er offentlige ville det sandsynligvis gøre denne teknologi til en reel fremtidsmulighed for valgende, og sikre de demokratiske processor langt ud i fremtiden.

STOPSTOP

\newpage
\appendix
\section{Bilag}
\subsection{OOP-kode}
\subsubsection{Transactions}
\lstinputlisting[language=Python]{../objects/transaction.py}
\newpage
\subsubsection{Blocks}
\lstinputlisting[language=Python]{../objects/blocks.py}
\newpage
\subsubsection{Blockchains}
\lstinputlisting[language=Python]{../objects/chain.py}
\subsubsection{Wallets}
\lstinputlisting[language=Python]{../objects/wallet.py}
\newpage
\subsection{Webserver-kode}
\subsubsection{Flask-app}
\lstinputlisting[language=Python]{../flaskApp/mainFlask.py}

\newpage
\printbibliography


\end{document}
